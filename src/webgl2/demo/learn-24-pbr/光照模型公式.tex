
物理模型光照公式

Lo(p,ωo)=∫Ω (kd * (c / π) + ks * (DFG / (4(ωo⋅n) * (ωi⋅n)))) * Li(p,ωi) * n⋅ωi * dωi
Lo(p,ωo)=∫Ω (kd * (c / π) + DFG / (4(ωo⋅n) * (ωi⋅n))) * Li(p,ωi) * n⋅ωi * dωi


Lo(p, ωo):  表示从点 p 出射方向为 ωo 的辐射度;
Ω:  所有可能的入射方向 ωi 的立体角;

kd: 折射辐射量
ks: 反射辐射量
c:  反照率(Albedo)纹理为每一个金属的纹素(Texel)(纹理像素)指定表面颜色或者基础反射率。这和我们之前使用过的漫反射纹理相当类似,
    不同的是所有光照信息都是由一个纹理中提取的。漫反射纹理的图像当中常常包含一些细小的阴影或者深色的裂纹,而反照率纹理中是不会有这些东西的。
    它应该只包含表面的颜色(或者折射吸收系数)。
ωo: 视角
ωi: 光角度
dωi 是对入射方向 ωi 的微分立体角。
Li(p,ωi): 光对于p点的总辐射度( 从点 p 入射方向为 ωi 的辐射度;)
  公式: lightColor * atte
    lightColor: 光辐射总量
    atte: 损耗度,一般为 (1 / pow(dis, 2)), dis为光位置与p点的距离

D:  法线分布函数, 从统计学上近似地表示了与某些(半程)向量h取向一致的微平面的比率。举例来说,假设给定向量h,
    如果我们的微平面中有百分之35与向量h取向一致,则法线分布函数或者说NDF将会返回0.35。
    公式:
      NDFGGXTR(n,h,α)=pow(α, 4) / (π * pow((pow(n⋅h, 2) * (pow(α, 4) −1) +1), 2))
      α 表示表面粗糙度取值[0, 1]
      n 表示法线
      h 表示半程向量
F:  菲涅尔方程描述的是在不同的表面角下表面所反射的光线所占的比率。菲涅尔(发音为Freh-nel)方程描述的是被反射的光线对比光线被折射的部分
    所占的比率,这个比率会随着我们观察的角度不同而不同。当光线碰撞到一个表面的时候,菲涅尔方程会根据观察角度告诉我们被反射的光线所占的百
    分比。利用这个反射比率和能量守恒原则,我们可以直接得出光线被折射的部分以及光线剩余的能量。
    公式:
      FSchlick(h,v,F0)=F0+(1−F0) * pow(1−(h⋅v), 5)
      h 表示半程向量
      v 表示视角
      F0 表示平面的基础反射率,它是利用所谓折射指数(Indices of Refraction)或者说IOR计算得出的。预计算出平面对于法向入射的结果(F0,
        处于0度角,好像直接看向表面一样),然后基于相应观察角的Fresnel-Schlick近似对这个值进行插值,用这种方法来进行进一步的估算。
        就是在视角与法线夹角为0度时的反射率,在图形学一般使用插值获取。
        公式:
          F0(albedo, metalness) = mix(F, albedo, metalness);
          F 一般是 0.04
          albedo: 表示基础反射率(0°视角反射率)
          metalness: 表示金属度
G:  微平面间相互遮蔽的比率,从几何函数从统计学上近似的求得了微平面间相互遮蔽的比率,这种相互遮蔽会损耗光线的能量。即反射光被去他曲面挡住的比例。
    公式:
      GSchlickGGX(n,v,l,k)=Gsub(n,v,k) * Gsub(n,l,k)
        Gsub(a,b,k) = a⋅b / (a⋅b * (1 − k)+k)
        n: 法线
        v: 视角
        l: 光角度
        k: α(表面粗糙度)的重映射,有两种公式,取决于针对直接光照还是针对IBL光照
          Kdirect=pow(α + 1, 2) / 8
          KIBL=pow(α, 2) / 2