
物理模型光照公式
Lo(p,ωo)=∫Ω (kd * (c / π) + ks * (DFG / (4(ωo⋅n) * (ωi⋅n)))) * Li(p,ωi) * n⋅ωi * dωi
将漫反射与反射分离
Lo(p,ωo)=∫Ω kd * (c / π) * Li(p,ωi) * n⋅ωi * dωi
将半球积分计算改为航向角ϕ和倾斜角θ积分计算 
Lo(p,ϕ,θ)=∫[2π, ϕ=0] ∫[π/2, θ=0] kd * (c / π) * Li(p,ϕ,θ) * cosθ * sinθ * dϕdθ
别给每个球坐标轴指定离散样本数量 n1和 n2以求其黎曼和,积分式会转换为以下离散版本:
Lo(p,ϕ,θ)= kd * (c / π) * (2π / n1) * (π / (2 * n2)) * Li(p,ϕ,θ) * cosθ * sinθ
         = kd * ((c * π) /(n1 * n2)) * Li(p,ϕ,θ) * cosθ * sinθ
由于计算量大,可以预计算环境量,变化公式
         = kd * c * ( (π * Li(p,ϕ,θ) * cosθ * sinθ) / (n1 * n2) )
采用环境贴图的方式,存储每个向量的值,再给模型使用。
  (Li(p,ϕ,θ) * cosθ * sinθ) * π / (n1 * n2)


  // 解决环境变量没有半程向量方法
  vec3 FSchlickRoughness(float dotNV, vec3 c, float m, float a) {
    vec3 F0 = mix(vec3(F0M), c, m);
    vec3 cons = max(vec3(1.0 - a), F0);
  
    return F0 + (cons - F0) * pow(1.0 - dotNV, 5.0);
  }
  